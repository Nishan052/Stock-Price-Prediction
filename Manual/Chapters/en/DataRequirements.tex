%%%%%%%%%%%%
%
% $Autor: Wings $
% $Datum: 2019-03-05 08:03:15Z $
% $Pfad: DataRequirements.tex $
% $Version: 4250 $
% !TeX spellcheck = en_GB/de_DE
% !TeX encoding = utf8
% !TeX root = manual 
% !TeX TXS-program:bibliography = txs:///biber
%
%%%%%%%%%%%%

\chapter{Data Requirements}

\section{Integrated Data Access}
The Stock Predictor application retrieves stock market data for the National Stock Exchange Fifty (NIFTY 50) index directly from a real-time financial data provider. This eliminates the need for users to manually upload datasets or prepare data files.

\subsection{Data Source}
Historical and current stock prices are automatically fetched using a live connection to Yahoo Finance. This ensures that the application always operates on up-to-date information without requiring any user intervention. The system downloads data such as:

\begin{itemize}
	\item Opening and closing prices of the NIFTY 50 index.
	\item Daily high and low values.
\end{itemize}

\subsection{Date Coverage}
The application supports predictions for any trading day from October 2007 up to the current date. Users may select either a past date (for backtesting) or a near-future date (e.g., tomorrow) to forecast the opening and closing prices.

\section{No Manual Data Handling}
\subsection{No File Upload Required}
Unlike traditional forecasting tools, this application does not require users to upload any Comma-Separated Values (CSV) files. All data ingestion is managed in the backend, ensuring that the correct format, column structure, and completeness are maintained automatically.

\subsection{Automated Preprocessing}
Once the data is fetched from the remote source, the application performs internal preprocessing to prepare it for prediction. This includes:

\begin{itemize}
	\item Filtering for valid trading days (excluding weekends).
	\item Cleaning and aligning feature values (e.g., open, close, high, low, volume).
	\item Preparing input sequences for the forecasting models based on the most recent available data.
\end{itemize}

\section{Data Consistency and Quality}
All datasets used by the application adhere to consistent structures verified at runtime. This ensures that forecasting models such as AutoRegressive Integrated Moving Average (ARIMA) and Long Short-Term Memory (LSTM) receive clean, well-structured input for reliable performance. By automating the data layer, the platform significantly reduces the risk of user error and improves forecast accuracy.